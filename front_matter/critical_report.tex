\documentclass[shorttitlesize=55,tocstyle=ref-genre]{ees}

\begin{document}

\eesTitlePage

\eesCriticalReport{
  –    & –   & org     & Bass figures appear in the following movements (bars in parentheses): 1.1, 1.2, 1.5, 1.6 (1–16), 1.9, 2.1, 2.3 (1–13), 2.7 (18–56, 72–84, 112–120, 125, 129–132, 147–152), and 2.8 (1–19). The remaining bass figures were added by the editor. \\
  1.2  & 29   & T       & 8th \sixteenthNote\ in \B1: \flat c16 \\
  1.3  & 65   & vl 2    & grace note missing in \B1 \\
       & 69   & vl 2    & grace note missing in \B1 \\
       & 72   & vla     & 1st/2nd \quarterNote\ in \B1: \flat b2 (no tremolo repeat) \\
       & 99   & vl 2    & last \eighthNote\ in \B1: f′8 \\
       & 177  & org     & bar in \B1: \flat e4. \\
  1.4  & 11   & A       & 1st \sixteenthNote\ in \B1: f′16 \\
       & 12   & vl 2    & last \sixteenthNote\ in \B1: g′16 \\
       & 18   & vl 1, 2 & 4th \sixteenthNote\ in \B1: f″16 \\
       & 23   & vl 2    & 4th \quarterNote\ in \B1: \flat b′16–c″16–d″16–\flat e″16 \\
       & 24   & A       & 4th \eighthNote\ in \B1: g′16–g′16 \\
       & 25   & T       & 4th \eighthNote\ in \B1: f′8 \\
       & 28   & A       & 8th \sixteenthNote\ missing in \B1 \\
  1.6  & 2    & vl 2    & 3rd \quarterNote: grace note missing in \B1 \\
       & 3    & fl 1    & 3rd \quarterNote: grace note missing in \B1 \\
       & 7    & vl 2    & 1st \quarterNote: grace note missing in \B1 \\
       & 12   & fl 1    & 4th \eighthNote: grace note missing in \B1 \\
       & 13   & fl 2    & 1st \quarterNote\ in \B1: \sharp c″8.–b′32–a′32 \\
       & 18   & vl 2    & 3rd \quarterNote: grace note missing in \B1 \\
       & 18   & vla     & bar in \B1: e′4–e4–a4–\crotchetRest \\
       & 21ff & T       & This rhythm is written as 16–8.–32 throughout the Aria\newline (bars 21, 22, 29, 44, 45, 47, 48, 56, and 83). \\
       & 26   & vla     & 3rd \eighthNote\ in \B1: b8 \\
       & 35   & vl 2    & 1st \quarterNote: grace note missing in \B1 \\
       & 37   & vl 1    & 2nd \quarterNote: grace note missing in \B1 \\
       & 41   & vl 2    & 1st \quarterNote: grace note missing in \B1 \\
       & 44   & vl 2    & 1st half of bar in \B1: e″8–d″16–\sharp c″16–\sharp c″8–b′16–a′16 \\
       & 48   & fl 2    & 1st half of bar in \B1: \sharp c″8–b′16.–a′32–a′8–\sharp g′16–b′32 \\
       & 50   & vl 2    & 2nd half of bar missing in \B1 \\
       & 58   & fl 1    & 3rd \eighthNote\ in \B1: a′8 \\
       & 60   & vl 2    & 1st \quarterNote: grace note missing in \B1 \\
       & 61   & vl 2    & 1st \quarterNote: grace note missing in \B1 \\
       & 74   & vla     & 2nd \quarterNote\ in \B1: d′8–\sharp d′8 \\
       & 77   & vla     & 6th \eighthNote\ in \B1: e′8 \\
       & 82   & vl 1    & 6th \eighthNote\ in \B1: a′8 \\
       & 98   & vl 1, 2 & last \quarterNote\ in \B1: a′16–e′16–c′16–a16 \\
  1.7  & 5    & vl 1    & last \eighthNote\ in \B1: a″8 \\
       &  7   & org     & 2nd half of bar in \B1: \sharp c4.–d8 \\
       & 15   & A       & 1st \quarterNote\ in \B1: b′4 \\
       & 16   & vl 1, s & last \eighthNote\ in \B1: \sharp f″8 \\
       & 16   & vl 2    & last \eighthNote\ in \B1: \sharp c″8 \\
       & 18   & S       & 3rd \quarterNote\ in \B1: \sharp f″4 \\
       & 19   & vl 1    & last \eighthNote\ in \B1: \sharp f″8 \\
       & 19   & vl 2    & last \eighthNote\ in \B1: \sharp c″8 \\
       & 19   & vla     & last \eighthNote\ in \B1: a′8 \\
       & 21   & vla     & bar in \B1: b′4–\quaverRest–a′8–a′4.–b′8 \\
  1.9  & 7    & vla     & 2nd half of bar in \B1: a′8–a′8–a′8–a′8 \\
       & 30   & vla     & bar in \B1: a′1 \\
       & 50   & org     & 2nd half note in \B1: \flat B2 \\
  1.11 & 22   & S       & 2nd half note in \B1: g′2 \\
  2.1  & 10   & org     & 2nd half note in \B1: \flat a2 \\
       & 14   & T       & 8th \sixteenthNote\ in \B1: b′16 \\
       & 24   & T       & 3rd \quarterNote\ in \B1: e′8.–e′16 \\
  2.2  & 1f   & vl 2    & in \B1 unison with vl 1 \\
       & 10   & org     & bar in \B1: d2. \\
       & 14   & ob 2    & grace note missing in \B1 \\
       & 14   & vl 2    & 2nd/3rd \quarterNote\ in \B1: g′4–\sharp f′4 \\
       & 22   & vl 2    & in \B1 unison with vl 1 \\
       & 25f  & vl 2    & in \B1 unison with vl 1 \\
       & 26   & ob 2    & 2nd/3rd \quarterNote\ in \B1: g′4–\sharp f′4 \\
       & 42   & ob 2    & grace note missing in \B1 \\
       & 46   & ob 2    & grace note missing in \B1 \\
       & 47   & vl 2    & 2nd/3rd \quarterNote\ in \B1: b′8–a′8–a′8–g′8 \\
       & 63   & vla     & 1st \quarterNote\ in \B1: d′8–b′8 \\
       & 68–71 & vl 2   & in \B1 unison with vl 1 \\
       & 71   & vl 2    & 3rd \quarterNote\ in \B1: a′4 \\
       & 81   & ob 2    & 1st note in \B1: \sharp c″8 \\
       & 81   & ob 2, vl 2 & 2nd/3rd \quarterNote\ in \B1: d″4–\sharp c″4 \\
       & 83   & ob 2    & grace note missing in \B1 \\
       & 84f  & vl 2    & in \B1 unison with vl 1 \\
       & 85   & ob 2    & grace note missing in \B1 \\
       & 101  & ob 2    & bar in \B1: grace c″4–b′2. \\
       & 108  & vl 2    & grace note missing in \B1 \\
       & 112  & ob 1    & 2nd/3rd \quarterNote\ in \B1: g″4–\sharp f″4 \\
       & 112  & vl 2, S 2 & 2nd/3rd \quarterNote\ in \B1: g′4–\sharp f′4 \\
       & 112  & S 1     & 2nd/3rd \quarterNote\ in \B1: grace a′4–g′2 \\
       & 113  & S 1     & 1st \quarterNote\ in \B1: \sharp f′4 \\
       & 123  & vla     & 1st \quarterNote\ in \B1: \sharp f′4 \\
       & 128  & vla     & bar in \B1: e′2–\crotchetRest \\
       & 130  & ob 2, vl 2 & 2nd/3rd \quarterNote\ in \B1: g′4–\sharp f′4 \\
       & 130  & S 2     & 1st \eighthNote: grace note missing in \B1 \\
       & 138  & ob 2    & 2nd/3rd \quarterNote\ in \B1: g′4–\sharp f′4 \\
       & 145–148 & vl 2 & in \B1 unison with vl 1 \\
       & 148  & vla     & bar in \B1: g′4–\crotchetRest–\crotchetRest \\
       & 150  & vl 1    & 1st \quarterNote\ in \B1: c″4 \\
       & 157f & vl 2    & in \B1 unison with vl 1 \\
       & 158  & ob 2    & 2nd/3rd \quarterNote\ in \B1: g′4–\sharp f′4 \\
       & 165  & vl 2    & in \B1 unison with vl 1 \\
       & 166  & ob 2    & 2nd/3rd \quarterNote\ in \B1: g′4–\sharp f′4 \\
       & 166  & vl 2    & 2nd/3rd \quarterNote\ in \B1: grace b′4–a′2 \\
       & 168  & ob 2    & 2nd/3rd \quarterNote\ in \B1: g′4–\sharp f′4 \\
       & 195f & vl 2    & in \B1 unison with vl 1 \\
       & 215  & vla     & bar in \B1: a2. \\
       & 219  & ob 2, S 2 & grace note missing in \B1 \\
       & 223  & S 1     & bar in \B1: \flat b′2–\sharp c″4 \\
       & 226  & S 2     & grace note missing in \B1 \\
       & 234  & org     & bar in \B1: d2. \\
       & 237f & vl 2    & in \B1 unison with vl 1 \\
       & 238  & ob 2    & 2nd/3rd \quarterNote\ in \B1: g′4–\sharp f′4 \\
       & 242  & ob 2    & 2nd/3rd \quarterNote\ in \B1: g′4–\sharp f′4 \\
  2.3  & 5    & vl, vla & bar missing in \B1 \\
  2.4  & 7    & vl 2    & last \sixteenthNote\ in \B1: a′16 \\
       & 8    & vl 2    & last \sixteenthNote\ in \B1: a′16 \\
       & 12   & vla     & bar in \B1: \sharp f′4.–\sharp f′8–\sharp f′4 \\
       & 23   & vla     & 3rd \quarterNote\ in \B1: a′16–g′16–\sharp f′16–e′16 \\
       & 44   & B       & 3rd \quarterNote\ in \B1: g4 \\
       & 67   & cor 2   & 1st \quarterNote\ in \B1: a′4 \\
       & 67   & vla     & 3rd \quarterNote\ in \B1: \sharp f′4 \\
       & 74   & vl 2    & grace note missing in \B1 \\
       & 85   & vla     & bar in \B1: e′4–\crotchetRest–\crotchetRest \\
       & 113  & vl 1, 2 & last \sixteenthNote\ in \B1: g′16 \\
       & 163  & cor 1   & 1st \quarterNote\ in \B1: g″4 \\
       & 176  & B       & last \sixteenthNote\ in \B1: b16 \\
       & 195  & B       & bar in \B1: B2–r4 \\
  2.5  & 2    & vl 2    & grace note missing in \B1 \\
       & 23   & A       & 7th \eighthNote\ missing in \B1 \\
       & 28   & org     & 1st \eighthNote\ in \B1: d8 \\
       & 29   & vl 2    & 3rd \quarterNote\ in \B1: \sharp f′4 \\
  2.6  & 32   & T       & bar in \B1: b1–b2 \\
       & 42   & T       & last half note in \B1: d′2 \\
  2.7  &  7   & vl 2    & c″ instead of \sharp b′ in \B1 \\
       & 97   & B       & 2nd \eighthNote\ in \B1: a8 \\
       & 98   & B       & 3rd \eighthNote\ in \B1: \sharp c′8 \\
       & 130  & A       & 3rd \quarterNote\ in \B1: \sharp f′4 \\
       & 130  & B       & last \sixteenthNote\ in \B1: e′16 \\
       & 131  & A       & last \quarterNote\ in \B1: \sharp f′4 \\
       & 133  & org     & 4th \eighthNote\ in \B1: \sharp c8 \\
       & 141  & B       & 4th to 6th \eighthNote\ in \B1: b8–b8–d′16–\sharp c′16 \\
       & 154  & vla     & 3rd \quarterNote\ in \B1: \sharp g′4 \\
  2.8  & 6    & vl 1    & 3rd \eighthNote\ in \B1: d″8 \\
       & 6    & A       & grace note missing in \B1 \\
       & 9    & vla     & 3rd \eighthNote\ in \B1: \sharp f′8 \\
       & 11   & vl 1    & grace note missing in \B1 \\
       & 12   & ob 2    & grace note missing in \B1 \\
       & 12   & vl 2    & grace note missing in \B1 \\
       & 12   & vla     & grace note missing in \B1 \\
       & 16   & vl 1    & 6th/7th \eighthNote\ in \B1: \sharp b″8–\sharp b″8 \\
       & 16   & vl 2    & 6th \eighthNote\ in \B1: b′8 \\
       & 20   & vl 1    & 1st \quarterNote\ in \B1: \quaverRest–d′8 \\
       & 21   & ob 2    & grace note missing in \B1 \\
       & 21   & vl 2    & grace note missing in \B1 \\
       & 23   & ob 2    & 1st half of bar in \B1: c″2 \\
       & 26   & S       & 1st half of bar in \B1: \quaverRest–f′8.–f′16–f′8;\newline 2nd half of bar: grace note missing in \B1 \\
       & 26   & A       & grace note missing in \B1 \\
       & 28   & vl 2    & grace note missing in \B1 \\
       & 30   & vl 1    & 7th \eighthNote\ in \B1: d″8 \\
       & 36   & vl 1    & 7th \eighthNote\ in \B1: a″32–a″32–a″32–a″32 \\
       & 37   & A       & 2nd \quarterNote\ in \B1: \sharp c″8–\sharp c″8 \\
       & 45   & B       & 3rd \quarterNote\ in \B1: f8–f8 \\
       & 46   & S       & 2nd \quarterNote\ in \B1: e″8.–f″16 \\
       & 46   & B       & 1st \quarterNote\ in \B1: g8–g8 \\
       & 47   & vla     & 2nd \quarterNote\ in \B1: a′8–g′8 \\
       & 49   & vl 2    & last \eighthNote\ in \B1: b′8 \\
}

\eesToc{
\part{ersteabtheylung}

\begin{movement}{jesudeine}
  \item[Coro]
  Jeſu, deine Pasſion\\
  will ich jetzt bedenken.\\
  Wolleſt mir vom Himmelsthron\\
  Geiſt und Andacht ſchenken.\\
  In dem Bild jetzund erſchein,\\
  Jeſu, meinem Herzen,\\
  wie du, unſer Heil zu ſein,\\
  litteſt alle Schmerzen.
\end{movement}

\begin{movement}{owelchein}
  \item[Tenore]
  O welch ein kläglich Bild,\\
  worin mein Jeſus mir erſcheinet:\\
  Er zagt und zittert,\\
  Tode’s Angſt erfüllt ſein Herz.\\
  Sein Auge weinet, weint blutgen Himmel,\\
  weint, und blutger Schweiß fließt in die Thränen.\\
  Du ſchauervolleſte der Szenen, Gethſemane.\\
  Mein Vater, ſpricht er jtzt,\\
  mein Vater, ich weiß, du kanſt es.\\
  Gieb, daß dieſer Kelch vorübergehe.\\
  Jedoch mein Wille nicht, der deinige geſchehe.\\
  So betet er und merkt,\\
  daß ihn ein Engel Gottes ſtärkt.\\
  Indesſen ſind die Jünger an Kidron eingeſchlafen.\\
  Jeſus Finger berührt ſie ſanft.\\
  Er ſpricht: Der Geiſt iſt willig,\\
  euer Leib nur nicht.\\
  Wacht, meine Lieben, wacht und betet.
\end{movement}

\begin{movement}{heiligerauch}
  \item[Tenore]
  Heiliger, auch ich bin Erde.\\
  Dieſer ſchwere Theil von Erde,\\
  dieſer Endlichkeit Gefühl\\
  drückt auch meine Seele nieder,\\
  wie ſie durch Gebeth und Lieder\\
  ſich zu dir erheben will.\\
  Schau zum Endlichen herunter,\\
  mache du die Seele munter,\\
  taufe ſie mit deinem Feuer,\\
  daß ſie kühner, daß ſie freyer\\
  ſich dem Chor der Himmel nahn,\\
  in ihr Loblied ſtammlen kan.
\end{movement}

\begin{movement}{meinherz}
  \item[Coro]
  Mein Herz iſt bereit, Gott,\\
  mein Herz iſt bereit,\\
  daß ich singe und lobe.\\
  (\bibleverse{Ps}(57/56:8))
\end{movement}

\begin{movement}{verraether}
  \item[Tenore]
  Verräther! Wie, dir muß es doch gelingen?\\
  Ach Gott, ich höre ſchon, ich höre Waffen klingen.\\
  Du kömmſt, mit dir der Mörder freche Schaar.\\
  Ach Gott, nun bringen ſie den Heilgen zu den Todten.\\
  Doch ſehet: Ohne Furcht ſtellt ſich der Stärkre dar.\\
  Er ſpricht ein Wort, da ſtürzen ſie zu Boden,\\
  da liegen ſie betäut und wie die Todten.\\
  Gelinde ruft er jezt: Ihr Feigen,\\
  dies iſt die Gewalt der Finſterniß und eure Stunde.\\
  Den ich muß ihn trinken, den Kelch,\\
  den mir mein Vater reicht,\\
  ſonſt kont ich leicht mehr als zwölf Legionen Engel winken.\\
  Und nun wird er gebunden, fort geführt zum Kaiphas.\\
  Nur Petrus folget ihm, gerührt von Mitleid,\\
  aber kurz iſt ſeine Treue.\\
  Er ſchwört, er kenne dieſen Menſchen nicht.\\
  Der Gottmenſch ſieht ihn an, indem ers ſpricht.\\
  Und Petrus ſtockt, und eine Zähre voller Reue\\
  netzt des Verräthers Angeſicht.
\end{movement}

\begin{movement}{lieblichfliesst}
  \item[Tenore]
  Lieblich, lieblich fließt die Zähre,\\
  holde Tugend, dir zur Ehre,\\
  die der Reue Wangen füllt.\\
  Weint, Verbrecher! Eure Sünden\\
  können einen Tilger finden.\\
  Des Mitlers ſanftes Herze\\
  ſchmilzt bey eurem heißen Schmerze,\\
  und ſein Zürnen wird geſtilt.
\end{movement}

\begin{movement}{wohldem}
  \item[Coro]
  Wohl dem, dem die Uebertretungen vergeben ſind,\\
  wohl dem, dem die Sünde bedecket iſt.\\
  (\bibleverse{Ps}(32/31:1))
\end{movement}

\begin{movement}{ichfalle}
  \item[Coro]
  Ich falle dir, mein Gott, zu Füßen,\\
  ich falle dir in deinen Arm.\\
  Ich komm mit wahrer Reu und Buße,\\
  ich ſchrei um Glauben, ach erbarm!\\
  Erbarme dich bey meiner Schuld\\
  und habe doch mit mir Gedult!
\end{movement}

\begin{movement}{erdessen}
  \item[Tenore]
  Er, desſen Allmachts Ruf\\
  der Weltenheer aus nichts erſchuff,\\
  er, Jeſus, wird, o Liebe,\\
  wird ein Spott der Sünder\\
  und ihrer Mordluſt Raub.\\
  Sie fordern ungeſtüm ſein Blut,\\
  ſein Blut kom über uns und unſre Kinder.\\
  So wühten alle.\\
  Blut entfließet ihm drauf ſtrohmweis.\\
  Gleich den Löwen, gleich jungen Löwen\\
  fallen ſie ihn an.\\
  Zum Hohn muß ihn ein Purpurkleid umgeben.\\
  Der Pöbel beugt vor ihm ſein Knie\\
  und läſtert ihm mit ſtolzer Müh.\\
  Seht, welch ein Menſch, ſpricht ſelbſt der Richter,\\
  der nie ſo viel gefühlt.\\
  Doch mehr entflamt ihr Grim,\\
  beredter wird der Mund der Böſewichter\\
  ſtürmt, bis ihn Pontius zum Tod am Creutz verdammt.

  \item[Alto]
  Seht, Chriſten, welch ein Menſch:\\
  Vorhin war er ſo ſchön,\\
  war tauſenden erkohren,\\
  dem feinſten Golde gleich ſein Haupt.\\
  Jtzt iſt ihm aller Hoheit Glanz geraubt,\\
  mit Dornen ſieht man es durchbohren.\\
  Sein Auge, daß uns ſanfte Luſt gebahr,\\
  worin ein Blick, ein Blick in Eden war,\\
  iſt geiſtlos, todesquahl zu ſehen, nur noch offen.\\
  Und ſeine Wangen und ſein Mund,\\
  die lauter Süßigkeiten troffen,\\
  wie Roſen und wie Mirhentropfen,\\
  ſind nun von Schlägen aufgeſchwellt und wund.\\
  Ach, ſeine ganze liebliche Geſtalt,\\
  vorhin wie Libanon, wie Zedern auserwehlet,\\
  iſt Elend und vor trauren alt,\\
  den er wird, ach, zu ſehr, zu ſehr gequälet.
\end{movement}

\begin{movement}{eristum}
  \item[Coro]
  Er iſt um unſere Misſethat willen ſo verwundet,\\
  und um unſere Sünden willen ſo zerſchlagen.\\
  (\bibleverse{Isa}(53:5))
\end{movement}

\begin{movement}{schreibedeine}
  \item[Coro]
  Schreibe deine blutgen Wunden\\
  mir, Herr, in mein Herz hinein,\\
  daß ſie mögen alle Stunden\\
  bey mir unvergeßen ſeyn.\\
  Du biſt doch mein liebſtes Guth,\\
  da mein ganzes Herze ruht.\\
  Laß mich hier zu deinen Füßen\\
  deine Lieb und Gunſt genießen.
\end{movement}

\part{zweyteabtheylung}

\begin{movement}{nunist}
  \item[Tenore]
  Nun iſt die feyerliche Stunde des großen Opfers da.\\
  Nun wirds erwürgt, das Lam, erwürgt am Golgatha.\\
  Wie viel mußt du für unſre Schulden am Creuz,\\
  Verſöhner Gottes, dulden?\\
  Da hängt er, ſeine Hand und Füßen ſind durchgraben.\\
  Ach, ein ganzes Labyrinth von Qualen iſt um ſeine Seele gewebt.\\
  O flöge doch ſein Herz aus ſeiner Höhle.\\
  Da hängt er! Seht, ſein Kleid in Blut getaucht!\\
  Seht, Frevler, deren Odem Rach und Unſin haucht.\\
  Wie Stimmen großer Waßer und ſtarker Donner\\
  rauſcht die Stimme ſeiner Haßer,\\
  auf ihrer Stirne glüth Verderben.\\
  Todt und Hölle öffnen ihre Rachen.\\
  Aber Jeſus ruft: Vergib es, vergib es ihnen, Vater.
\end{movement}

\begin{movement}{gottam}
  \item[Soprano 1, 2]
  Gott am Creutze, lehre mich,\\
  meinen Bruder ſanft begegnen.\\
  Gott am Creutze, lehre mich,\\
  dir gleich meine Feinde ſeegnen.\\
  O! wie himmliſch, lehrts dein Herz.\\
  Wehe denen, welche nicht\\
  ihrem Bruder gern verzeihen.\\
  Gott wird einſten im Gericht\\
  ihnen wieder nicht verzeihen.\\
  Ach, ihr Lohn iſt Ewger Schmertz.
\end{movement}

\begin{movement}{esschweben}
  \item[Tenore]
  Es ſchweben Seraphim von fern\\
  am Schädelrollen Hügel.\\
  Tief ſtaunend über ihren Herrn\\
  bedecken ſie mit ihren Flügeln das Antlitz.\\
  Jeſus leidet, ſagen ſie,\\
  ja was des Menſchen Seele nie gedacht hat, leidet er.\\
  Allein, ſo bitter ſein Schmerz iſt,\\
  will er doch die edelſte der Mütter, Maria, Troſt ertheilen,\\
  und befiehlet jetzo dies ſeinem Liebling an,\\
  der Liebling ſchüzt Maria.\\
  Jeſus wird erheitert und verkündigt drauf einen Sünder,\\
  den ſein Glaub entſündigt, Erbarmung,\\
  und der Seel Unſterblichkeit.\\
  Ich ſage, ſpricht er, dir, du wirſt noch heut\\
  im Paradieſe mit mir ſeyn.
\end{movement}

\begin{movement}{hoerts}
  \item[Basso]
  Hörts, Chriſten, das iſt unſer Glaube,\\
  die Seele reißt ſich aus dem Staube,\\
  ihr Weſen iſt Unſterblichkeit.\\
  Heil allen, die ſich höhers Leben\\
  durch eine ſchöne That erſtreben:\\
  Euch Hoffenden wirds gegeben nach dieſer Zeit.
\end{movement}

\begin{movement}{meineseele}
  \item[Coro]
  Meine Seele dürſtet nach Gott,\\
  nach dem lebendigen Gott.\\
  Ach! Wenn werde ich dahin kommen,\\
  daß ich Gottes Angeſicht ſchaue?\\
  (\bibleverse{Ps}(42/41:3))
\end{movement}

\begin{movement}{ofreud}
  \item[Coro]
  O! Freud, o! Luſt, o! Leben,\\
  o! göldnes Hauß, o! ſchönſte Zier.\\
  Wir wollen kräftig ſtreben\\
  in dieſer Sterblichkeit nach dir.\\
  O! Gottes Antlitz ſehen,\\
  o! ſtets in Friede ſeyn.\\
  O! bey den Engeln ſtehen,\\
  o! theurer Himmelsſchein.\\
  O! Herrlichkeit ohn Ende,\\
  mein Gott, wenn dirs gefällt,\\
  ſo nimm mich auf behende.\\
  Nun gute Nacht, o! Welt.
\end{movement}

\begin{movement}{duschoener}
  \item[Tenore]
  Du ſchöner Morgenſtern, wie biſt du doch ſo tief\\
  von deinem Himmel herab geſunken.\\
  Ach, ſie dauret noch und wächſt,\\
  die Qual des Leidenden,\\
  Jehovens Schrecken ſind wider ihn\\
  in Schlachtordnung geſtellt.\\
  Die Pfeile des Allmächtigen,\\
  ſie ſtecken jtzt all in ſeiner Bruſt,\\
  die Hölle fällt auf ihn.\\
  Er ruft: Warum haſt du mich, Gott, verlaßen?\\
  Und ſo, von Gott verlaßen,\\
  büßt er die Sünden einer ganzen Welt.\\
  Noch einmal röchelt er: Mich dürſtet.\\
  Aber ſehet, den lezten Trunk, den er erflehet,\\
  vermiſchen ſie mit Galle.\\
  Größres Leid war nicht zu denken.\\
  Jeſus ſpricht: Es iſt vollbracht.\\
  Mein Vater, ich befehle in deine Hände meine Seele.\\
  Und neigt ſein Haupt und ſtirbt.

  \item[Basso]
  Jeſus Chriſtus ſtirbt, ſo klagen alle Himmel.\\
  Verhül, o Sonne, dich in Flohr,\\
  beb, Erde, macht, ihr Felſen, ein Getümmel,\\
  ihr Todten Gottes, kömmt aus eurer Gruft hervor!\\
  Kömmt, ſchaut, was jtzt geſchieht!\\
  Erſtaunt, fühlt, zittert, ſchaut:\\
  Die ganze Schöpfung ächze laut,\\
  denn Jeſus Chriſtus ſtirbt.

  \item[Sop., Tenore]
  Mein Jeſus ſtirbt, ihr Augen weint,\\
  ach, weinet um den Menſchenfreund.\\
  Ach, er, der Lehren Gottes gab,\\
  ſinkt in des Todes Nacht hinab.

  \item[Basso]
  Seyd getroſt, ihr Weinenden:\\
  Tod und Hölle ſind nun überwunden\\
  durch des Lammes Bluth.\\
  Darum ſey fröhlich, Erde,\\
  freuet euch, ihr Himmel und die darinnen wohnen.

  \item[Alto]
  Ihr Myriaden, die am Throne Gottes knien!\\
  Warum verſtummen eure Pſalmen?\\
  Warum werft ihr ſie hin, die Kronen und die Palmen?\\
  Ach, ſterben, ſterben ſeht ihr ihn?\\
  Drum ſchweigen eure Pſalmen?\\
  Es ſchweigt der Spähren Harmonie,\\
  den König Salems klagen ſie.\\
  Im Himmel und auf Erden herſchet nur ein Wille,\\
  ein lautes Schrecken erſt, und dann,\\
  dann trauren ſie und werden ſtille.

  \item[Sop., Tenore]
  Mein Jeſus ſtirbt, wie furchtbar groß\\
  war ſeine Qual, wie namenslos.\\
  Er ſtirbt am Creuz am Golgatha,\\
  den Tod der Knechte ſtirbt er da.

  \item[Basso]
  Seyd getroſt, ihr Weinenden:\\
  Tod und Hölle ſind nun überwunden\\
  durch des Lammes Bluth.\\
  Darum ſey fröhlich, Erde,\\
  freuet euch, ihr Himmel und die darinnen wohnen.

  \item[Alto]
  Ein Gottmenſch ſtirbt, für Sünder blutet er.\\
  Gedanke, wer begreift dich, wer?\\
  Groß biſt du, groß vor allen mächtigen Gedanken,\\
  in welchen jemals Seraphin verſanken.\\
  Kein Sterblicher vermag dich durchzuſchaun,\\
  und ſelbſt der Engel, den es lüſte durchzuſchaun,\\
  der bebt zurück, ihn überfällt ein heilig Graun.

  \item[Coro]
  Mein Jeſus ſtirbt, ihr Thränen fließt,\\
  er hat für uns, für uns gebüßt.

  \item[Basso]
  Seyd getroſt, ihr Weinenden:\\
  Tod und Hölle ſind nun überwunden\\
  durch des Lammes Bluth.

  \item[Coro]
  O wehe dem, der Sünde thut,\\
  ihn ſchrecke Jeſu theures Blut.

  \item[Basso]
  Seyd getroſt, ihr Weinenden:\\
  Tod und Hölle ſind nun überwunden\\
  durch des Lammes Bluth.\\
  Darum ſey fröhlich, Erde,\\
  freuet euch, ihr Himmel und die darinnen wohnen.

  \item[Coro]
  Dank, Preis und Ehre wollen wir ihm weihen,\\
  anbeten immer, und uns freuen.\\
  Dank, Preis und Ehre dem, der an dem Creuze ſtarb,\\
  und ewges Heyl erwarb.\\
  Hallelujah!
\end{movement}

\begin{movement}{versoehner}
  \item[Coro]
  Verſöhner, heilges Gottes Lamm,\\
  laß deinen Tod und deine Wunden,\\
  ach, laß ſie uns in unſer lezten Stunde\\
  Troſt für die Seele ſeyn.\\
  Sie bluten jtzt, bald ſtrahlen ſie,\\
  Gericht dem Sünder, Huld dem Frommen.\\
  O! Wolluſt! Wir werden nie\\
  in dein Gericht, Verſöhner, kommen:\\
  Dein Blut macht unſre Herzen rein.
\end{movement}
}

\eesScore

\end{document}
